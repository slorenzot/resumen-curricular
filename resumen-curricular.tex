% resume.tex
% vim:set ft=tex spell:

\documentclass[10pt,letterpaper]{article}
\usepackage[letterpaper,margin=0.75in]{geometry}
\usepackage[utf8]{inputenc}
\usepackage{mdwlist}
\usepackage[T1]{fontenc}
\usepackage{textcomp}

\pagestyle{empty}
\setlength{\tabcolsep}{0em}

% indentsection style, used for sections that aren't already in lists
% that need indentation to the level of all text in the document
\newenvironment{indentsection}[1]%
{\begin{list}{}%
	{\setlength{\leftmargin}{#1}}%
	\item[]%
}
{\end{list}}

% opposite of above; bump a section back toward the left margin
\newenvironment{unindentsection}[1]%
{\begin{list}{}%
	{\setlength{\leftmargin}{-0.5#1}}%
	\item[]%
}
{\end{list}}

% format two pieces of text, one left aligned and one right aligned
\newcommand{\headerrow}[2]
{\begin{tabular*}{\linewidth}{l@{\extracolsep{\fill}}r}
	#1 &
	#2 \\
\end{tabular*}}

% make "C++" look pretty when used in text by touching up the plus signs
\newcommand{\CPP}
{C\nolinebreak[4]\hspace{-.05em}\raisebox{.22ex}{\footnotesize\bf ++}}

% and the actual content starts here
\begin{document}

\begin{center}
{\LARGE \textbf{Soulberto Lorenzo Torres}}

Cantarrana, Villa Félix Barreto\ \ \textbullet
\ \ Casa Magyubeliz \ \textbullet
\ \ Cumaná, Edo Sucre 6101
\\
+58 (293) 467-1258 / +58 (412) 085-1147\ \ \textbullet
\ \ Correo-e: \textit{slorenzot@gmail.com} / Twitter: \textit{@slorenzot}
\end{center}

\hrule
\vspace{-0.4em}
\subsection*{Experiencia laboral}

\begin{itemize}
	\parskip=0.1em

	\item
	\headerrow
		{\textbf{Dirección de Información y Comunicación Corporativa (DICC)}}
		{\textbf{Cumaná, Sucre}}
	\\
	\headerrow
		{\emph{Desarrollador Web - Rectorado de la Universidad de Oriente}}
		{\emph{2002 -- 2003}}

	\item
	\headerrow
		{\textbf{FUNDACITE SUCRE}}
		{\textbf{Cumaná, Sucre}}
	\\
	\headerrow
		{\emph{Instructor - Programa Academia de Software Libre ASL}}
		{\emph{2007 -- 2008}}
	\begin{itemize*}
		\item Instructor de Usuario Final y herramiento ofimáticas
		\item Instructor del Nivel de Soporte Técnico
		\item Instructor del Nivel de Administradores
	\end{itemize*}
	
	\item
	\headerrow
		{\textbf{FUNDACITE SUCRE}}
		{\textbf{Cumaná, Sucre}}
	\\
	\headerrow
		{\emph{Planificador I - Coordinación de Telemática e Innovación Tecnológica}}
		{\emph{2008 -- 2009}}
	\begin{itemize*}
		\item Responsable del Programa Nacional Academia de Software Libre ASL sede Sucre
	\end{itemize*}

	\item
	\headerrow
		{\textbf{Zona Educativa del Estado Sucre}}
		{\textbf{Cumaná, Sucre}}
	\\
	\headerrow
		{\emph{Jefe de División - Informática y Sistemas}}
		{\emph{2009 -- 2012}}
	
	\item
	\headerrow
		{\textbf{Universidad Politécnica del Oeste del Estado Sucre "Clodosbaldo Russian"}}
		{\textbf{Cumaná, Sucre}}
	\\
	\headerrow
		{\emph{Analista programador - Oficina de Informática y Telemática}}
		{\emph{2012 -- Presente}}

\end{itemize}


\hrule
\vspace{-0.4em}
\subsection*{Educación}

\begin{itemize}
	\parskip=0.1em

	\item 
	\headerrow
		{\textbf{Universidad de Oriente - Núcleo de Sucre}}
		{\textbf{Cumaná, Edo. Sucre}}
	\\
	\headerrow
		{\emph{Coordinación del Programa de Licenciatura en Informática}}
		{\emph{1997 -- 2008}}
	\begin{itemize*}
		\item Licenciado en Informática.
	\end{itemize*}

\end{itemize}

\hrule
\vspace{-0.4em}
\subsection*{Cursos y talleres}

\begin{indentsection}{\parindent}
\hyphenpenalty=1000
  \item {\large 2009}
    \begin{itemize*}
	    \item Uso de Herramientas de Software Licre para Dispacidad Visual. Dictado por Mundo Accesible, 8 horas. MegaInfocentro, Cumaná Edo. Sucre, Diciembre de 2009.
    \end{itemize*}
  \item {\large 2013}
    \begin{itemize*}
	    \item Curso de cableado con Fibra Óptica. Laboratorio de Informática, UPTOS Clodosbaldo Russián.
    \end{itemize*}
    \begin{itemize*}
	    \item Curso de Nociones Básicas de Arduino. Laboratorio de Electrónica, UPTOS Clodosbaldo Russián, Cumaná, Edo. Sucre.
    \end{itemize*}
\end{indentsection}

\hrule
\vspace{-0.4em}
\subsection*{Eventos}

\begin{indentsection}{\parindent}
\hyphenpenalty=1000
  \item {\large 2002}
    \begin{itemize*}
	    \item Organizador del III Marathon de Programación ACM-UDO Sucre - Universidad de Oriente, Núcleo de Sucre, Cumaná Edo. Sucre.
    \end{itemize*}
  \item {\large 2007}
    \begin{itemize*}
	    \item Colaborador del 3er Congreso Nacional de Software Libre - Teatro Luís Mariano Rivera, Cumaná Edo. Sucre.
	    \item Participación en el V Foro Mundial del Conocimiento Libre - Ciudad Bolívar.
    \end{itemize*}
  \item {\large 2008}
    \begin{itemize*}
	    \item Colaborador en el Festival Latinoamericano de Instalación de Software Libre - Instituto Universitario de Tecnología Cumaná, Edo. Sucre
	    \item Ponente del XXXI Aniversario del Instituto Universitario de Tecnología Cumaná sede Cariaco - Edo. Sucre 
	    \item Ponente en el 4to Congreso Nacional de Softwar Libre sede Monagas  Maturín, Edo. Monagas
	    \item Colaborador en el 4to Congreso Nacional de Softwar Libre sede Sucre - Cumaná, Edo. Sucre
	    \item Ponente en el XV aniversario del Programa de Licenciatura en Informática - Universidad de Oriente - Núcleo de Sucre, Cumaná Edo. Sucre.
    \end{itemize*}
  \item {\large 2009}
    \begin{itemize*}
	    \item Colaborador en el Festival Latinoamericano de Instalación de Software Libre - Instituto Universitario de Tecnología Cumaná, Edo. Sucre
	    \item Ponente en el 5to Congreso Nacional de Softwar Libre sede Sucre - Cumaná, Edo. Sucre
	    \item Ponente en el I Foro Estadal de Tecnologías Libres Hacia la Implementación de Software Libre en la Administración Pública - Teatro Luís Mariano Rivera - Cumaná, Edo. Sucre
    \end{itemize*}
  \item {\large 2010}
    \begin{itemize*}
	    \item Colaborador en el Festival Latinoamericano de Instalación de Software Libre - Teatro Luís Mariano Rivera - Cumaná, Edo. Sucre
	    \item Juez evaluador en el VIII Encuentro Infantil de Informática Educativa - Auditorio IUT Cumaná, Cumaná Edo. Sucre
    \end{itemize*}
    \item {\large 2012}
    \begin{itemize*}
	    \item Colaborador en el Festival Latinoamericano de Instalación de Software Libre - Universidad Nacional Experimental de las Fuerzas Armadas UNEFA - Cumaná, Edo. Sucre
    \end{itemize*}
    \item {\large 2014}
    \begin{itemize*}
            \item Ponente en el Festival Latinoamericano de Instalación de Software Libre - Fundación para el Desarrollo de la Ciencia y la Tecnología del Estado Sucre - FUNDACITE SUCRE.
    \end{itemize*}
\end{indentsection}

\hrule
\vspace{-0.4em}
\subsection*{Áreas de interés}

\begin{indentsection}{\parindent}
\hyphenpenalty=1000
\begin{itemize*}
  \item Software Libre / Open Source
  \item Investigación y desarrollo de software
  \item Desarrollo y migración de sistemas
\end{itemize*}
\end{indentsection}

\hrule
\vspace{-0.4em}
\subsection*{Habilidades técnicas}

\begin{indentsection}{\parindent}
\hyphenpenalty=1000
\begin{description*}
  \item[Conocimientos:]
  \item - Implementación y administración de redes.
	\item - Instalación y administración de servicios en plataformas libres
	\item - Experiencia con diversas distribuciones GNU/Linux, especialmente Debian GNU/Linux
	\item - Conocimientos en Programación Orientada a Objetos y uso de patrones de diseño
	\item[Lenguajes:]
	C, PHP, Java, JavaScript, \LaTeX, Perl, Python, shell script, SQL
\end{description*}
\end{indentsection}

\hrule
\vspace{-0.4em}
\subsection*{Cualidades}

\begin{indentsection}{\parindent}
\hyphenpenalty=1000
\begin{description*}
	\item - Proactivo, responsable y puntual
	\item - Defensor insansable del movimiento Software Libre en la región
  \item - Promotor del uso de los sistemas operativos libres y del desarrollo de proyectos bajo la
  filosofía GNU
  \item - Excelente desempeño en sirtuaciones de mucha presión
  \item - Capacidad de trabajo bajo autoridad, en coordinación y en grupo	
\end{description*}
\end{indentsection}

\begin{center}
  {\small {\footnotesize Actualizado: octubre 2014}}
\end{center}

\end{document}









